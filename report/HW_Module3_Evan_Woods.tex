\documentclass[12pt]{article}
\usepackage{amsmath}
\usepackage{graphicx}
\usepackage{fancyhdr}
\usepackage{hyperref}
\usepackage{listings}
\usepackage{times}

% Page Setup
\usepackage[a4paper, margin=1in]{geometry}

% Font Settings
\renewcommand{\rmdefault}{ptm} % Times New Roman font

% Spacing
\linespread{1.0} % Single spacing

% Set up the header and footer
\pagestyle{fancy}
\fancyhf{}
\fancyhead[L]{6.419x Module 3}
\fancyhead[C]{Written Report}
\fancyhead[R]{Your Name}

% Title Section
\title{Written Report – 6.419x Module 1}
\author{Evan Woods}
\date{}

\begin{document}

\maketitle

% Table of Contents
\tableofcontents
\newpage


\noindent \textbf{Name:} Evan Woods \\
\noindent \textbf{Module:} 6.419x Module 3

\section{Part 1: Thesis}

\subsection{Question 1.1}
\textit{Clearly states a sociological question which is interesting and relevant to the data. The question must be sociologically motivated: for example, “Compare the network structure in 2003 vs 2009" is not a good question, without further context. If you have some reason to believe that the network structure changes in those years, then you should make that your central question: for example, “Did crimes involving youth offenders become more organized and structured over the years" is a better question, from which comparing the structure in different years becomes part of the methodology to answer the question. More examples of possible questions for cooffending networks are provided below. }

Parkinson's is a disease that affects the social and cognitive abilities of people who have the diagnosis. In this body of work, I sought to identify which patients in a dataset are either cognitively normal, under mild cognitive impairment (hereafter refered to as MCI), or have Parkinson's disease dementia (hereafter refered to as PDD) based on computed fractional Amplitude Low-Frequency Fluctuations (hereafter refered to as fALFF) from the mean of the Blood Oxidation Level Dependent (hereafter refered to as BOLD) signal of the Substantia Nigra extracted from their functional Magnetic Resonance Image (hereafter refered to as fMRI). I furthered this analysis by seeking to answer the question: of the patients who have been identified to belong to the same diagnosis, what regions of the brain have no statistically significant difference of correlations of fALFF values? That is to say, what regions of the brain are similar in BOLD activity in patients with a common diagnosis, how does the activity of these patients differ between the diagnosed groups, and what are possible remedies to induce healthy levels of cognition determined by the fALFF of the BOLD signal in patients diagnosed with MCI or PDD to achieve an fALFF of the BOLD signal that resembles that of the patients diagnosed to be cognitively normal?   

\section{Part 2: Methods}

\subsection{Question 2.1}
\textit{ (2 points) Describes methodology for network analysis.}

The data was collected as an OpenNeuro dataset on Parkinson's Disease, functional connectivity, and cognition from https://openneuro.org/datasets/ds004392/versions/1.0.0 \cite{ds004392:1.0.0}. In the dataset, all patients were diagnosed with Parkinson's, scanned with resting state fMRI, and underwent a neurocognitive test battery. Of these patients, there are three sub-diagnoses of patients who are either cognitively normal, have mild cognitive impairment, or have Parkinson's disease dementia. The patients were given scores of their attention, executive function, global cognition, language cognition, and visuospatial cognition based as determined by the results of the neurocognitive test battery. The eleven tests included a total score on teh Montreal Cognitive Assessment, a Mattis Dementia Rating Scale 2, a Trail Making Test part A, a Trail Making Test part B, a Brief Test of Attention, an oral Symbol Digits Modality Test, a Boston Naming Test, a California Verbal Learning Test (2nd edition with the total learning score calculated from the 5 total trials), another California Verbal Learning Test (2nd edition with a long delay free recall total score), a F-A-S Verbal Phonemic Fluency test, and a Judgement of Line Orientation test. All units of the metrics which resulted from the tests were standardized with a norm of 0 and a variance of 1.

fALFF values have been indicative of cognitive function. A study titled "Default mode network mediates low‐frequency fluctuations in brain activity and behavior during sustained attention" indicates found that "...low‐frequency fluctuations were significantly increased in the (Default Mode Network) but not in the (Anterior Nucleus) during sustained attention..." \cite{35903957}. This information was leveraged to assert fALFF as a key indicator metric of cognitive ability in the data.

To calculate the fALFF scores of the BOLD signal from the fMRI data, the BOLD signal would need to first be derived from the fMRI data, and the fALFF calculated from this signal per patient per region. Once the fALFF scores were gathered, spectral clustering of the scaled fALFF of the Substantia Nigra region implementing 3 KMeans clusters with a nearest neighbors affinity as hyperparameters would create label predictions of the data. The integrity of the predicted clusters was verified by calculating the silhouette score of the labels and scaled data as well as calculating the convex Hull Ratio of the hull volume of the Convex Hull Volume of the scaled fALFF score of the substantia nigra per the data volume of the product of the peak-to-peak values of the scaled fALFF of the substantia nigra. The scaled fALFF scores of the substantia nigra were reduced to the first two principal components with principal component analysis (hereafter refered to as PCA), before a non-linear T-Stochastic Neighbor Embedding (hereafter refered to as T-SNE) further transformed the data so as clear clustering would be identified on a plot. These cluster labels where then presumed as ground truth diagnoses for the patients. 




\subsection{Question 2.2}
\textit{(2 points) Grader is convinced that the methodology makes sense for the question to be answered. Grader is convinced that no additional methodology within the bounds of techniques taught and discussed in this module could be applied beyond what was described. The grader should only consider additional methodology that adds meaningfully to the answer for the question: additions that simply repeat or confirm the presented results should not be considered by the grader. If a justification is provided for why a particular method was not used, the grader should be convinced by that argument. }

Solution: Write your answer in a brief and clear language. In addition, you should add all materials that you have consulted to in the Reference section at the end of the report. These materials could be a paper [1], a book [2], or some internet materials [3].

\section{Part 3: Results}

\subsection{Question 3.1}
\textit{(2 points) Presents results, including figures and/or statistics, which address the question of interest.}

Solution: Write your answer in a brief and clear language. In addition, you should add all materials that you have consulted to in the Reference section at the end of the report. These materials could be a paper [1], a book [2], or some internet materials [3].

\subsection{Question 3.2}
\textit{(2 points) The described methodology has been applied in complete and the results shown (that is, the author did not forget to include anything they discussed in the methodology.) }

Solution: Write your answer in a brief and clear language. In addition, you should add all materials that you have consulted to in the Reference section at the end of the report. These materials could be a paper [1], a book [2], or some internet materials [3].

\section{Part 4: Discussion}
\subsection{Question 4.1}
Solution: Write your answer in a brief and clear language. In addition, you should add all materials that you have consulted to in the Reference section at the end of the report. These materials could be a paper [1], a book [2], or some internet materials [3].

\subsection{Question 4.2}
Solution: Write your answer in a brief and clear language. In addition, you should add all materials that you have consulted to in the Reference section at the end of the report. These materials could be a paper [1], a book [2], or some internet materials [3].

\subsection{Question 4.3}
Solution: Write your answer in a brief and clear language. In addition, you should add all materials that you have consulted to in the Reference section at the end of the report. These materials could be a paper [1], a book [2], or some internet materials [3].

\newpage


\begin{center}
\begin{lstlisting}[language=Python]
# This is a sample Python code snippet
def add_numbers(a, b):
    return a + b
result = add_numbers(5, 3)
print(result)
\end{lstlisting}
\end{center}

For longer code segments, they should be placed in the Appendix after the main text.

% Insert References Section
\bibliographystyle{plain} % Citation style (alternatives: abbrv, unsrt, apalike, IEEEtran, etc.)
\bibliography{references} % Use the .bib file (without the .bib extension)

\end{document}
