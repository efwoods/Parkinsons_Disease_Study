\documentclass[12pt]{article}
\usepackage{amsmath}
\usepackage{graphicx}
\usepackage{fancyhdr}
\usepackage{hyperref}
\usepackage{listings}
\usepackage{times}

% Page Setup
\usepackage[a4paper, margin=1in]{geometry}

% Font Settings
\renewcommand{\rmdefault}{ptm} % Times New Roman font

% Spacing
\linespread{1.0} % Single spacing

% Set up the header and footer
\pagestyle{fancy}
\fancyhf{}
\fancyhead[L]{6.419x Module 3}
\fancyhead[C]{Written Report}
\fancyhead[R]{Your Name}

% Title Section
\title{Written Report – 6.419x Module 1}
\author{Evan Woods}
\date{}

\begin{document}

\maketitle

% Table of Contents
\tableofcontents
\newpage


\noindent \textbf{Name:} Evan Woods \\
\noindent \textbf{Module:} 6.419x Module 3

\section{Part 1: Thesis}

\subsection{Question 1.1}
\textit{Clearly states a sociological question which is interesting and relevant to the data. The question must be sociologically motivated: for example, “Compare the network structure in 2003 vs 2009" is not a good question, without further context. If you have some reason to believe that the network structure changes in those years, then you should make that your central question: for example, “Did crimes involving youth offenders become more organized and structured over the years" is a better question, from which comparing the structure in different years becomes part of the methodology to answer the question. More examples of possible questions for cooffending networks are provided below. }

Solution: Write your answer in a brief and clear language. In addition, you should add all materials that you have consulted to in the Reference section at the end of the report. These materials could be a paper [1], a book [2], or some internet materials [3].


\section{Part 2: Methods}

\subsection{Question 2.1}
\textit{ (2 points) Describes methodology for network analysis.}
Solution: Write your solution here.

\subsection{Question 2.2}
\textit{(2 points) Grader is convinced that the methodology makes sense for the question to be answered. Grader is convinced that no additional methodology within the bounds of techniques taught and discussed in this module could be applied beyond what was described. The grader should only consider additional methodology that adds meaningfully to the answer for the question: additions that simply repeat or confirm the presented results should not be considered by the grader. If a justification is provided for why a particular method was not used, the grader should be convinced by that argument. }

Solution: Write your answer in a brief and clear language. In addition, you should add all materials that you have consulted to in the Reference section at the end of the report. These materials could be a paper [1], a book [2], or some internet materials [3].

\section{Part 3: Results}

\subsection{Question 3.1}
\textit{(2 points) Presents results, including figures and/or statistics, which address the question of interest.}

Solution: Write your answer in a brief and clear language. In addition, you should add all materials that you have consulted to in the Reference section at the end of the report. These materials could be a paper [1], a book [2], or some internet materials [3].

\subsection{Question 3.2}
\textit{(2 points) The described methodology has been applied in complete and the results shown (that is, the author did not forget to include anything they discussed in the methodology.) }

Solution: Write your answer in a brief and clear language. In addition, you should add all materials that you have consulted to in the Reference section at the end of the report. These materials could be a paper [1], a book [2], or some internet materials [3].

\section{Part 4: Discussion}
\subsection{Question 4.1}
Solution: Write your answer in a brief and clear language. In addition, you should add all materials that you have consulted to in the Reference section at the end of the report. These materials could be a paper [1], a book [2], or some internet materials [3].

\subsection{Question 4.2}
Solution: Write your answer in a brief and clear language. In addition, you should add all materials that you have consulted to in the Reference section at the end of the report. These materials could be a paper [1], a book [2], or some internet materials [3].

\subsection{Question 4.3}
Solution: Write your answer in a brief and clear language. In addition, you should add all materials that you have consulted to in the Reference section at the end of the report. These materials could be a paper [1], a book [2], or some internet materials [3].

\newpage

\begin{center}
\begin{lstlisting}[language=Python]
# This is a sample Python code snippet
def add_numbers(a, b):
    return a + b
result = add_numbers(5, 3)
print(result)
\end{lstlisting}
\end{center}

For longer code segments, they should be placed in the Appendix after the main text.

\section*{References}

\begin{enumerate}
    \item R. L. Wasserstein and N. A. Lazar, “The ASA statement on p-values: context, process, and purpose,” The American Statistician, vol. 70, no. 2, pp. 129-133, 2016.
    \item B. Gustavii, \textit{How to write and illustrate a scientific paper}, Cambridge University Press, 2017.
    \item Wikipedia, “Principal component analysis,” Accessed: Sep. 2021. [Online]. Available: \url{https://en.wikipedia.org/wiki/Principal_component_analysis}
\end{enumerate}

\end{document}
