\documentclass[12pt]{article}
\usepackage{amsmath}
\usepackage{graphicx}
\usepackage{fancyhdr}
\usepackage{hyperref}
\usepackage{listings}
\usepackage{times}

% Page Setup
\usepackage[a4paper, margin=1in]{geometry}

% Font Settings
\renewcommand{\rmdefault}{ptm} % Times New Roman font

% Spacing
\linespread{1.0} % Single spacing

% Set up the header and footer
\pagestyle{fancy}
\fancyhf{}
\fancyhead[L]{6.419x Module 1}
\fancyhead[C]{Written Report}
\fancyhead[R]{Your Name}

% Title Section
\title{Written Report – 6.419x Module 1}
\author{Your Name}
\date{}

\begin{document}

\maketitle

\noindent \textbf{Name:} Your Name \\
\noindent \textbf{Module:} 6.419x Module 1

\section*{Problem 1.1}

\subsection*{1. (2 points) \textit{How would you run a randomized controlled double-blind experiment to determine the effectiveness of the vaccine? Write down procedures for the experimenter to follow. (Maximum 200 words)}}

Solution: Write your answer in a brief and clear language. In addition, you should add all materials that you have consulted to in the Reference section at the end of the report. These materials could be a paper [1], a book [2], or some internet materials [3].

\subsection*{2. (3 points) \textit{For each of the NFIP study, and the Randomized controlled double-blind experiment above, which numbers (or estimates) show the effectiveness of the vaccine? Describe whether the estimates suggest the vaccine is effective. (Maximum 200 words)}}

Solution: Write your solution here.

\section*{Problem 1.3}

\subsection*{(a-1). (2 points) \textit{Your colleague on education studies really cares about what can improve the education outcome in early childhood. He thinks the ideal planning should be to include as much variables as possible and regress children's educational outcome on the set. Then we select the variables that are shown to be statistically significant and inform the policy makers. Is this approach likely to produce the intended good policies?}}

Solution: Write your solution here.

\subsection*{(a-2). (3 points) \textit{Your friend hears your point, and think it makes sense. He also hears about that with more data, relations are less likely to be observed just by chance, and inference becomes more accurate. He asks, if he gets more and more data, will the procedure he proposes find the true effects?}}

Solution: Write your solution here.

\section*{Problem 1.5}

\subsection*{Code Example}

Here is an example of how to format a small code snippet (less than 10 lines) centered within the main text. You can change the code in the following block to suit your needs:

\begin{center}
\begin{lstlisting}[language=Python]
# This is a sample Python code snippet
def add_numbers(a, b):
    return a + b
result = add_numbers(5, 3)
print(result)
\end{lstlisting}
\end{center}

For longer code segments, they should be placed in the Appendix after the main text.

\section*{References}

\begin{enumerate}
    \item R. L. Wasserstein and N. A. Lazar, “The ASA statement on p-values: context, process, and purpose,” The American Statistician, vol. 70, no. 2, pp. 129-133, 2016.
    \item B. Gustavii, \textit{How to write and illustrate a scientific paper}, Cambridge University Press, 2017.
    \item Wikipedia, “Principal component analysis,” Accessed: Sep. 2021. [Online]. Available: \url{https://en.wikipedia.org/wiki/Principal_component_analysis}
\end{enumerate}

\end{document}
